\documentclass[t,
aspectratio=169,  % Auskommentieren für einfaches 4:3 Verhältnis
%handout,         % Aktivieren um Overlays zu ignorieren
]{beamer}
\usepackage[utf8]{inputenc}
\usepackage[T1]{fontenc}
\usepackage[ngerman]{babel}
\usepackage{libertine}
\usepackage{pgfpages}

\usepackage{booktabs}
\usepackage{relsize}
\usepackage{hyperref}
\renewcommand*{\UrlFont}{\ttfamily\smaller\relax}

\title{Folientitel}
\subtitle{Ein einführendes Beispiel\\ Zweite, und\\ dritte Zeile}
\author[A. Fischer]  % Kurzer Name für Fußzeile
       {Andreas Fischer \\  % Voller Name für Titelfolie
	\href{mailto:andreas.fischer@th-deg.de}{andreas.fischer@th-deg.de}}

%\usetheme{thdeg} % Altes Design
\usetheme[thd]{thdeg2022} % Optionen sind Fakultäten: biw, bwl, et-mt, mb, nuw, agw, ai, ecri

\begin{document}

\begin{frame}
\titlepage{}
\end{frame}

\section{Beispiele}

\begin{frame}
\frametitle{Eine Folienüberschrift}
\framesubtitle{Ein Folien-Untertitel}

\begin{block}{Ein Block}
	\begin{itemize}
		\item Mehrere
		\item Items
		\begin{itemize}
			\item Sogar in
			\item zweiter Reihe
		\end{itemize}
	\end{itemize}
\end{block}

\ldots
\end{frame}

\begin{frame}{Eine Folie mit zweispaltigem Layout}
	\begin{columns}[T]
	\begin{column}[T]{.45\linewidth}
		\begin{block}{Linker Teil}
			\ldots
		\end{block}
	\end{column}
	\begin{column}[T]{.45\linewidth}
		\begin{block}{Rechter Teil}
			\ldots
		\end{block}
	\end{column}
	\end{columns}
\end{frame}

\begin{frame}{Eine Folie\ldots}
	\begin{itemize}[<+->]
		\item auf\ldots
		\item der\ldots
		\item alles\ldots
		\item erst\ldots
		\item nacheinander\ldots
		\item erscheint.
	\end{itemize}
\end{frame}

\section{Mehr Beispiele}

\begin{frame}{Eine Grafik}
	\begin{center}
	\begin{tikzpicture}
		\node[draw,circle] (A) {A};
		\node[draw,circle,right=of A] (B) {B};
		\draw<+->[bend left,-latex] (A) to node[midway,above] {\tiny TikZ Grafiken sind} (B);
		\draw<+->[bend left,-latex] (B) to node[midway,below] {\tiny schon integriert} (A);
	\end{tikzpicture}
	\end{center}
\end{frame}

\begin{frame}{}
	\begin{itemize}
		\item Eine Folie
		\item ohne Überschrift
	\end{itemize}
\end{frame}



\end{document}
